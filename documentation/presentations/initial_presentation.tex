\documentclass[10pt,tgadventor, onlymath]{beamer}

\usepackage{graphicx,amsmath,amssymb,tikz,psfrag,neuralnetwork}

\input defs.tex
\graphicspath{ {./figures/} }

%% formatting

\mode<presentation>
{
\usetheme{default}
\usecolortheme{seahorse}
}
\setbeamertemplate{navigation symbols}{}
\usecolortheme[rgb={0.03,0.28,0.59}]{structure}
\setbeamertemplate{itemize subitem}{--}
\setbeamertemplate{frametitle} {
	\begin{center}
	  {\large\bf \insertframetitle}
	\end{center}
}

\newcommand\footlineon{
  \setbeamertemplate{footline} {
    \begin{beamercolorbox}[ht=2.5ex,dp=1.125ex,leftskip=.8cm,rightskip=.6cm]{structure}
      \footnotesize \insertsection
      \hfill
      {\insertframenumber}
    \end{beamercolorbox}
    \vskip 0.45cm
  }
}
\footlineon

\AtBeginSection[] 
{ 
	\begin{frame}<beamer> 
		\tableofcontents[currentsection,currentsubsection] 
	\end{frame} 
} 


\tikzstyle{state}=[shape=circle,draw=blue!30,fill=blue!10]
\tikzstyle{observation}=[shape=rectangle,draw=orange!30,fill=orange!10]
\tikzstyle{lightedge}=[<-, dashed]
\tikzstyle{mainstate}=[state, thick]
\tikzstyle{mainedge}=[<-, thick]
\tikzstyle{block} = [draw,rectangle,thick,minimum height=2em,minimum width=2em]
\tikzstyle{sum} = [draw,circle,inner sep=0mm,minimum size=2mm]
\tikzstyle{connector} = [->,thick]
\tikzstyle{line} = [thick]
\tikzstyle{branch} = [circle,inner sep=0pt,minimum size=1mm,fill=black,draw=black]
\tikzstyle{guide} = []
\tikzstyle{snakeline} = [connector, decorate, decoration={pre length=0.2cm,
                         post length=0.2cm, snake, amplitude=.4mm,
                         segment length=2mm},thick, magenta, ->]



%% begin presentation

\title{\large \bfseries Power Allocation in Heterogeneous Networks for Base Stations with Multiple Antennas}

\author{Peter Hartig\\[3ex]
}

\date{\today}

\begin{document}

\frame{
\thispagestyle{empty}
\titlepage
}

\section{System Description}
\begin{frame}
\frametitle{The Heterogeneous Network}
TODO get image of network
\end{frame}

\begin{frame}
\frametitle{Players of the Game}


Individual femto cell base stations (FC-BS) are the players of this game.
\\
Femto Cells are characterized by the following parameters
\begin{itemize}
\item 
	Each FC-BS  $f \in \{1 ... F\}$ is considered to have a number of antennas $T_f$ with which to transmit to $K_f$ femto cell users. It is assumed throughout the remainder that $T_f \geq K_f$.
\\
\item 
	FC-BSs with multiple antennas ($T_f >1$) can beamform their transmission using the precoding 	
	matrix $\mathbf{U}_{\mathrm{f}} \in \mathbb{C}_{T_f \times K_f}$ such that the transmitted 		
	signal is $\mathbf{s}_{\mathrm{f}
	}= \mathbf{U_{\mathrm{f}}}\mathbf{x_{\mathrm{f}}}$. Here $\mathbf{x_{\mathrm{f}}}$ is the 		
	normalized vector of symbols for users of FC-BS $f$ (i.e. $E[\|\mathbf{x}_{\mathrm{f}}
	\|_2^2] \; f \in \{1 ... F\}$).
\\
\item 
	FC-BS $f$ has power constraint $trace(\mathbf{U}_f^H\mathbf{U}_f) \leq P^{Total}_{f} $.
\\
\item
	 FC-BSs are assumed to be spaced far apart in distance. Therefore, FC-BS $f$ can be modeled as 
	 causing no interference to the users of FC-BS $j \in \{1 ... F\}\backslash f$
\item 
	FC-BSs are assumed to have a utility function $U_f()$ based upon the quality of service 		
	provided to its users. TODO (Discuss Reasonable functions)
\\
\item 
	FC-BS $f$ is assumed to know the downlink channel ($\mathbf{H_\mathrm{f}}$) from its transmission 		
	antennas to all served users.
% TODO(Simulate degradation with incomplete CSI solution?)
\\
\end{itemize}

\end{frame}

\begin{frame}
\frametitle{The Heterogeneous Network}
\end{frame}

\begin{frame}
\frametitle{Objectives}
\begin{enumerate}
\item Find a NE between all players
\item Find a solution that can be distributed
\item Find a solution which minimizes communication overhead in system

\end{enumerate}
\end{frame}

\section{Decoding Environment}



\section{Initial Results}


\end{document}
